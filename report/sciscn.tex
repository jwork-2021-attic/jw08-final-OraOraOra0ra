%%%%cls文件默认改用ctexart类,如果使用使用cctart类,请使用xelatex并修改SCIS2022cn.cls文件对应内容;
% !TEX encoding = UTF-8
% !TEX program = pdflatex
%-----------------------------------------------------------------------
% 中国科学: 信息科学 中文模板, 请用 TexLive 编译
% http://scis.scichina.com
%-----------------------------------------------------------------------

\documentclass{SCIS2022cn}
%%%%%%%%%%%%%%%%%%%%%%%%%%%%%%%%%%%%%%%%%%%%%%%%%%%%%%%
%%% 作者附加的定义
%%% 常用环境已经加载好, 不需要重复加载
%%%%%%%%%%%%%%%%%%%%%%%%%%%%%%%%%%%%%%%%%%%%%%%%%%%%%%%


%%%%%%%%%%%%%%%%%%%%%%%%%%%%%%%%%%%%%%%%%%%%%%%%%%%%%%%
%%% 开始
%%%%%%%%%%%%%%%%%%%%%%%%%%%%%%%%%%%%%%%%%%%%%%%%%%%%%%%
\begin{document}

%%%%%%%%%%%%%%%%%%%%%%%%%%%%%%%%%%%%%%%%%%%%%%%%%%%%%%%
%%% 作者不需要修改此处信息
\ArticleType{论文}
%\SpecialTopic{}
%\Luntan{中国科学院学部\quad 科学与技术前沿论坛}
\Year{2022}
\Vol{52}
\No{1}
\BeginPage{1}
\DOI{}
\ReceiveDate{2021-xx-xx}
\ReviseDate{2021-xx-xx}
\AcceptDate{2021-xx-xx}
\OnlineDate{2022-xx-xx}
\AuthorMark{孙文戈}
%\AuthorCitation{作者1, 作者2, 作者3, 等}
%\enAuthorCitation{Xing M, Xing M M, Xing M, et al}
%%%%%%%%%%%%%%%%%%%%%%%%%%%%%%%%%%%%%%%%%%%%%%%%%%%%%%%

\title{Developing a Java Game from Scratch}

\entitle{Title}{Title for citation}

\author[1]{孙文戈}{Ming XING1}{}

%若英文部分的emaillist太长需要换行的话,形式单独写在这里
%\enauthoremaillist{xingming1@xxxx.xxx, xingming2@xxxx.xxx, xingming3@xxxx.xxx, xingming4\\@xxxx.xxx, xingming5@xxxx.xxx}

%\comment{\dag~同等贡献}
%\encomment{\dag~Equal contribution}

\address[1]{南京大学计算机科学与技术系, 南京 210023}{Affiliation, City {\rm 000000}, Country}

\Foundation{国家自然科学基金 (批准号: 0000000, 0000000, 00000000)}

\abstract{本文有关Java高级程序设计课程第八次作业,作为葫芦娃游戏大作业的报告。本次游戏主要基于曹春老师教授的面向对象思想,使用JavaFX设计图形界面,并使用到java.io与java.nio完成存档与网络对战的设计。下文将介绍游戏有关的设计理念与问题,并对于本学期所学课程进行总结。}

\enabstract{An abstract (about 200 words) is a summary of the content of the manuscript. It should briefly describe the research purpose, method, result and conclusion. The extremely professional terms, special signals, figures, tables, chemical structural formula, and equations should be avoided here, and citation of references is not allowed.}

\keywords{Java,葫芦娃,游戏,面向对象}
\enkeywords{keyword1, keyword2, keyword3, keyword4, keyword5}

\maketitle

\section{开发目标}

本次大作业主要有三个部分,分别是基于jw04的迷宫任务改造为一个ruguelike的葫芦娃与妖精两方对战游戏,同时实现多线程;实现游戏保存功能,使用maven管理项目以及编写junit单元测试;实现网络对战。为实现以上要求,我主要基于jw04的作业,运用面向对象的知识,学习JavaFX用法代替之前的ASCII Panel实现GUI页面,使用IO完成游戏存档,NIO实现网络对战。

下面是关于我的游戏具体介绍与其灵感来源。

%\newpage

\subsection{游戏介绍}

本次作业我写的游戏名为The story of the hulus,即讲述葫芦娃们的故事。主要剧情为蛇精诱骗和葫芦娃们重新划分山界(占格子游戏,即人物移动会将走过的格子染色,人物只能走自己的颜色的格子与未被走过的白色格子),并在结束之后用剑攻击所有的葫芦娃,使得全地图仍为蛇精所有(全部恢复为蛇精的灰色)。同时在为游戏增添的多人模式中,将占格子游戏改为最高可以支持九人(贴图只找了九个,理论上限为地图大小)同时在线联机游玩。


\subsection{灵感来源}

不知道从哪里看到的一句话—“游戏的意义在于讲一个好的故事”,所以此次我想通过作业来写一个有剧情的游戏,来讲述一个蛇精耍阴谋诡计打倒葫芦娃们的故事。

其实最开始我的目标并不是这个。在作业提及继续完成游戏时,因为jw04获奖的经历,我想继续在游戏中加入能让人耳目一新的元素。我很自然地联想到拳皇这种经典格斗对战游戏,里面攻击攒能量释放绝招的画面引起我的注意。于是我最先的设想是,设计一个双人对战,在他们移动攻击的过程中会积攒能量条,能量条满释放绝招。释放绝招时我想使整个游戏窗口播放技能释放动画,如拳皇里放绝招的背景,圣斗士里天马流星拳的招式展示。当时就感觉这种想法设计出来展示时候会比较amazing吧。

但是由于迟迟没有行动,新的作业要求“多人对战”发布,让我不禁联想到一些以前我玩过的多人在线游戏,比如Pummel Party,一个集合各种趣味小游戏的派对型游戏。它里面集成了各种类型的小游戏,玩家们每回合会进行其中一项游戏进行竞技,根据每轮游戏得分最终得到的总分决出名次。由于许久没玩,比较容易改编成课程作业易于实现的我那时候只想到了占格子游戏(前文有介绍),当时觉得占格子实现起来应该比较简单,便初步把作业形式定为pummel party这种的集成式游戏,可以自由选择游戏模式(格斗与占格子)。

又是迟迟没有开始,也因为个人觉得单纯的将两个不相关游戏塞到一起供选择比较生硬,为此要取舍苦恼了一段时间。不过有天早上刷牙时我突然想到了可以以故事形式,情节场景转游戏场景实现不同游戏的切换来消除两个游戏间切换的生硬。于是本次作业所设计的游戏模式雏形诞生了!
%%%%%%%%%%%%%%%%%%%%%%%%%%%%%%%%%%%%%%%%%%%%%%%%%%%%%%%
%%% 致谢
%%% 非必选
%%%%%%%%%%%%%%%%%%%%%%%%%%%%%%%%%%%%%%%%%%%%%%%%%%%%%%%
%\Acknowledgements{致谢.}

%%%%%%%%%%%%%%%%%%%%%%%%%%%%%%%%%%%%%%%%%%%%%%%%%%%%%%%
%%% 补充材料说明
%%% 非必选
%%%%%%%%%%%%%%%%%%%%%%%%%%%%%%%%%%%%%%%%%%%%%%%%%%%%%%%
%\Supplements{补充材料.}

%%%%%%%%%%%%%%%%%%%%%%%%%%%%%%%%%%%%%%%%%%%%%%%%%%%%%%%
%%% 参考文献, {}为引用的标签, 数字/字母均可
%%% 文中上标引用: \upcite{1,2}
%%% 文中正常引用: \cite{1,2}
%%%%%%%%%%%%%%%%%%%%%%%%%%%%%%%%%%%%%%%%%%%%%%%%%%%%%%%
\begin{thebibliography}{99}

    \bibitem{1} Author A, Author B, Author C. Reference title. Journal, Year, Vol: Number or pages

    \bibitem{2} Author A, Author B, Author C, et al. Reference title. In: Proceedings of Conference, Place, Year. Number or pages

\end{thebibliography}

%%%%%%%%%%%%%%%%%%%%%%%%%%%%%%%%%%%%%%%%%%%%%%%%%%%%%%%
%%% 附录章节, 自动从A编号, 以\section开始一节
%%% 非必选
%%%%%%%%%%%%%%%%%%%%%%%%%%%%%%%%%%%%%%%%%%%%%%%%%%%%%%%
%\begin{appendix}
%\section{附录}
%附录从这里开始.
%\begin{figure}[H]
%\centering
%%\includegraphics{fig1.eps}
%\cnenfigcaption{附录里的图}{Caption}
%\label{fig1}
%\end{figure}
%\end{appendix}


%%%%%%%%%%%%%%%%%%%%%%%%%%%%%%%%%%%%%%%%%%%%%%%%%%%%%%%
%%% 自动生成英文标题部分
%%%%%%%%%%%%%%%%%%%%%%%%%%%%%%%%%%%%%%%%%%%%%%%%%%%%%%%
\makeentitle


%%%%%%%%%%%%%%%%%%%%%%%%%%%%%%%%%%%%%%%%%%%%%%%%%%%%%%%
%%% 补充材料, 以附件形式作网络在线, 不出现在印刷版中
%%% 不做加工和排版, 仅用于获得图片和表格编号
%%% 自动从I编号, 以\section开始一节
%%% 可以没有\section
%%%%%%%%%%%%%%%%%%%%%%%%%%%%%%%%%%%%%%%%%%%%%%%%%%%%%%%
%\begin{supplement}
%\section{supplement1}
%自动从I编号, 以section开始一节.
%\begin{figure}[H]
%\centering
%\includegraphics{fig1.eps}
%\cnenfigcaption{补充材料里的图}{Caption}
%\label{fig1}
%\end{figure}
%\end{supplement}

\end{document}


%%%%%%%%%%%%%%%%%%%%%%%%%%%%%%%%%%%%%%%%%%%%%%%%%%%%%%%
%%% 本模板使用的latex排版示例
%%%%%%%%%%%%%%%%%%%%%%%%%%%%%%%%%%%%%%%%%%%%%%%%%%%%%%%

%%% 章节
\section{}
\subsection{}
\subsubsection{}


%%% 普通列表
\begin{itemize}
    \item Aaa aaa.
    \item Bbb bbb.
    \item Ccc ccc.
\end{itemize}

%%% 自由编号列表
\begin{itemize}
    \itemindent 4em
    \item[(1)] Aaa aaa.
    \item[(2)] Bbb bbb.
    \item[(3)] Ccc ccc.
\end{itemize}

%%% 定义、定理、引理、推论等, 可用下列标签
%%% definition 定义
%%% theorem 定理
%%% lemma 引理
%%% corollary 推论
%%% axiom 公理
%%% propsition 命题
%%% example 例
%%% exercise 习题
%%% solution 解名
%%% notation 注
%%% assumption 假设
%%% remark 注释
%%% property 性质
%%% []中的名称可以省略, \label{引用名}可在正文中引用
\begin{definition}[定义名]\label{def1}
    定义内容.
\end{definition}



%%% 单图
%%% 可在文中使用图\ref{fig1}引用图编号
\begin{figure}[!t]
    \centering
    \includegraphics{fig1.eps}
    \cnenfigcaption{中文图题}{Caption}
    \label{fig1}
\end{figure}

%%% 并排图
%%% 可在文中使用图\ref{fig1}、图\ref{fig2}引用图编号
\begin{figure}[!t]
    \centering
    \begin{minipage}[c]{0.48\textwidth}
        \centering
        \includegraphics{fig1.eps}
    \end{minipage}
    \hspace{0.02\textwidth}
    \begin{minipage}[c]{0.48\textwidth}
        \centering
        \includegraphics{fig2.eps}
    \end{minipage}\\[3mm]
    \begin{minipage}[t]{0.48\textwidth}
        \centering
        \cnenfigcaption{中文图题1}{Caption1}
        \label{fig1}
    \end{minipage}
    \hspace{0.02\textwidth}
    \begin{minipage}[t]{0.48\textwidth}
        \centering
        \cnenfigcaption{中文图题2}{Caption2}
        \label{fig2}
    \end{minipage}
\end{figure}

%%% 并排子图
%%% 需要英文分图题 (a)...; (b)...
\begin{figure}[!t]
    \centering
    \begin{minipage}[c]{0.48\textwidth}
        \centering
        \includegraphics{subfig1.eps}
    \end{minipage}
    \hspace{0.02\textwidth}
    \begin{minipage}[c]{0.48\textwidth}
        \centering
        \includegraphics{subfig2.eps}
    \end{minipage}
    \cnenfigcaption{中文图题}{Caption}
    \label{fig1}
\end{figure}

%%% 算法
%%% 可在文中使用 算法\ref{alg1} 引用算法编号
\begin{algorithm}
    %\floatname{algorithm}{Algorithm}%更改算法前缀名称
    %\renewcommand{\algorithmicrequire}{\textbf{Input:}}% 更改输入名称
    %\renewcommand{\algorithmicensure}{\textbf{Output:}}% 更改输出名称
    \footnotesize
    \caption{算法标题}
    \label{alg1}
    \begin{algorithmic}[1]
        \REQUIRE $n \geq 0 \vee x \neq 0$;
        \ENSURE $y = x^n$;
        \STATE $y \Leftarrow 1$;
        \IF{$n < 0$}
        \STATE $X \Leftarrow 1 / x$;
        \STATE $N \Leftarrow -n$;
        \ELSE
        \STATE $X \Leftarrow x$;
        \STATE $N \Leftarrow n$;
        \ENDIF
        \WHILE{$N \neq 0$}
        \IF{$N$ is even}
        \STATE $X \Leftarrow X \times X$;
        \STATE $N \Leftarrow N / 2$;
        \ELSE[$N$ is odd]
        \STATE $y \Leftarrow y \times X$;
        \STATE $N \Leftarrow N - 1$;
        \ENDIF
        \ENDWHILE
    \end{algorithmic}
\end{algorithm}

%%% 简单表格
%%% 可在文中使用 表\ref{tab1} 引用表编号
\begin{table}[!t]
    \cnentablecaption{表题}{Caption}
    \label{tab1}
    \footnotesize
    \tabcolsep 49pt %space between two columns. 用于调整列间距
    \begin{tabular*}{\textwidth}{cccc}
        \toprule
        Title a & Title b & Title c & Title d \\\hline
        Aaa & Bbb & Ccc & Ddd\\
        Aaa & Bbb & Ccc & Ddd\\
        Aaa & Bbb & Ccc & Ddd\\
        \bottomrule
    \end{tabular*}
\end{table}

%%% 换行表格
\begin{table}[!t]
    \cnentablecaption{表题}{Caption}
    \label{tab1}
    \footnotesize
    \def\tabblank{\hspace*{10mm}} %blank leaving of both side of the table. 左右两边的留白
    \begin{tabularx}{\textwidth} %using p{?mm} to define the width of a column. 用p{?mm}控制列宽
        {@{\tabblank}@{\extracolsep{\fill}}cccp{100mm}@{\tabblank}}
        \toprule
        Title a & Title b & Title c & Title d                                                                                        \\\hline
        Aaa     & Bbb     & Ccc     & Ddd ddd ddd ddd.

        Ddd ddd ddd ddd ddd ddd ddd ddd ddd ddd ddd ddd ddd ddd ddd ddd ddd ddd ddd ddd ddd ddd ddd ddd ddd ddd ddd ddd ddd ddd ddd. \\
        Aaa     & Bbb     & Ccc     & Ddd ddd ddd ddd.                                                                               \\
        Aaa     & Bbb     & Ccc     & Ddd ddd ddd ddd.                                                                               \\
        \bottomrule
    \end{tabularx}
\end{table}

%%% 单行公式
%%% 可在文中使用 (\ref{eq1})式 引用公式编号
%%% 如果是句子开头, 使用 公式(\ref{eq1}) 引用
\begin{equation}
    A(d,f)=d^{l}a^{d}(f),
    \label{eq1}
\end{equation}

%%% 不编号的单行公式
\begin{equation}
    \nonumber
    A(d,f)=d^{l}a^{d}(f),
\end{equation}

%%% 公式组
\begin{eqnarray}
    \nonumber
    &X=[x_{11},x_{12},\ldots,x_{ij},\ldots ,x_{n-1,n}]^{\rm T},\\
    \nonumber
    &\varepsilon=[e_{11},e_{12},\ldots ,e_{ij},\ldots ,e_{n-1,n}],\\
    \nonumber
    &T=[t_{11},t_{12},\ldots ,t_{ij},\ldots ,t_{n-1,n}].
\end{eqnarray}

%%% 条件公式
\begin{eqnarray}
    \sum_{j=1}^{n}x_{ij}-\sum_{k=1}^{n}x_{ki}=
    \left\{
    \begin{aligned}
        1,  & \quad i=1,             \\
        0,  & \quad i=2,\ldots ,n-1, \\
        -1, & \quad i=n.
    \end{aligned}
    \right.
    \label{eq1}
\end{eqnarray}

%%% 其他格式
\footnote{Comments.} %footnote. 脚注
\raisebox{-1pt}[0mm][0mm]{xxxx} %put xxxx upper or lower. 控制xxxx的垂直位置

%%% 图说撑满
\Caption\protect\linebreak \leftline{Caption}
